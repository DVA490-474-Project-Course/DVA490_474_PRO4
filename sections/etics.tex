%===============================================================================
\section{Ethical and Societal Considerations}
\label{section:ethical_and_societal_considerations}

%-------------------------------------------------------------------------------
% Responsible students
%-------------------------------------------------------------------------------

\textbf{Responsible students: Carl Larsson}

%-------------------------------------------------------------------------------
% Ethics
%-------------------------------------------------------------------------------

% Type of ethics
Only robot ethical considerations with humans as the ethical subjects and the ethics of good society (as is defined in Coeckelbergh book Robot Ethics\:\cite{coeckelbergh_robot_2022}) is considered in this paper.
% Responsibility
Coeckelbergh\:\cite{coeckelbergh_robot_2022} raises the ethical consideration of who is responsible for a robot, and if it is ethical as a developer to just abandon the robot after it has been created, leaving all responsibilities behind. Coeckelbergh\:\cite{coeckelbergh_robot_2022} goes on to discuss the ethical dilemma of what type of robots could be considered, or not considered, ethical to create.
In this project the responsibility of the robots, their use, and the creation of these robots fall on the stakeholder. The developers have no choice in whether it is ethical to develop this kind of robots (the project must be done in order to pass the course). Furthermore, the developers are unable to follow along with their creation after the course ends, and can thus not be held responsible for the robot. Finally, the developers are not in charge of whether the projects findings and research is published or not.
% Loosing jobs
The research could result in robots competing with professional football players, possibly resulting in reducing the human football scene and professional players loosing their jobs. Entertainment, especially sport, is a prominent aspect in today's society, so should the previous consideration play out, then it could lead to a transformation of the sports entertainment. This would in turn have some consequences, most likely including economical, societal, social and more. Similar considerations are brought up by Coeckelbergh\:\cite{coeckelbergh_robot_2022}, however this paper specifically applied those considerations to this project.
% Applications
Multi agent research, and robots in general, could be applied to military and surveillance sectors. This in turn raises questions of privacy and harm to humans, and who is to be responsible should that happen. Additionally, like Coeckelbergh\:\cite{coeckelbergh_robot_2022} asks the reader to ponder, in the case that the robots are employed for surveillance then what if the robot does not respect human dignity? Military applications require serious considerations as well, the disconnect that a robot creates could be seen as making it easier to kill and start conflicts since it is easier to lack sympathy for a target when one never has to interact or see them. This can be further extended to the view of some \ac{ai} as "black boxes", the inability to know the exact workings of an \ac{ai} and what it will do, even the developers might not fully understand them. If the robot \ac{ai} harms someone or something that is should not have (it was not its intended function to do so), then who is to blame?

%===============================================================================

\begin{comment}
In the first instance, "ethics" refers to research ethics issues. Does your choice of research question or method imply any research ethics position? For example, if you are interviewing people for your work, can you guarantee their anonymity, and how will you use the information you get from them? Are there other ethical aspects to consider in your career? Might there be ethical aspects to the outcome of your job? You should indicate if you believe that your work does not contain any research ethics issues. 

You should also critically review and analyse your work concerning societal aspects. For example, you could discuss how your work relates to economic, social and environmental sustainability objectives. There may also be legal and political aspects to your work. 
\end{comment}

%===============================================================================